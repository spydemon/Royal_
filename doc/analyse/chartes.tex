\section{Chartes à respecter}

\subsection{Charte ergonomique}
Notre projet visant une utilisation privée il devra être facile d'utilisation afin d'être adapter au plus grand nombres.
De même comme nous améliorerons Royal afin de répondre aux demandes de notres projets et que le but du projet Royal était d'ameliorer le projet Birdy afin de faciliter son utilisation ainsi que la gestion des livres dans le logiciel.  

Pour ce qui est de l'application Android, nous devrons nous adapter à une utilisation tactile de l'application. 
C'est pourquoi les éléments de l'écran devrons être assez grand afin d'être manipulés simplement. 

\subsection{Charte graphique}
Pour l'apparence graphique de l'aplication Royal\_ nous veillerons à respecter le thème déja présent dans Royal afin d'être en accord avec ce qui existe déja.

L'application mobile nous permettra une plus grande liberté concernant l'apparence. 
Les fonds d'écran des différentes pages de l'application devrons être similaire afin de permettre une certaine homogénéité dans notre application.
Pour cette apparence nous choisirons tout d'abord un fond rouge s'adaptant avec le logo de Royal mais cette apparence pourra être modifée si elle ne correspond pas aux demandes du client au différentes remises de prototypes.

\subsection{Charte de programmation}

Comme nous reprennons Royal un logiciel existant, il nous faudra adapter notre style de programmation à celui déja présents dans le programme actuellement.
Tout d'abord en ce qui concerne les commentaires nous les écrirons en anglais comme c'était le cas dans Royal afin que le code soit réutilisable facilement par d'autres (vu qu'il s'agit d'un projet open source).

Nous utiliserons aussi les même attributs que ceux déja présents c'est à dire:

\begin{flushleft}
 Attributs obligatoires :
\end{flushleft}

    @return [détails]

    @param nomDeParam [qu'est-ce ?]

    @throws nomDeException pourquoi?

\begin{flushleft}
 Attributs facultatifs :
\end{flushleft}

    @see Classe

    @see Classe\#attribut

    @see Classe\#methode

    autres voir doc :  http://www.oracle.com/technetwork/java/javase/documentation/index-137868.html

\begin{flushleft}
 Afin d'eviter tout erreur de compilation au niveau de javac nous eviterons tout accent dans notre programme.
\end{flushleft}

\begin{flushleft}
 Pour les noms de variables nous commencerons chaque nom de variable par la première lettre de son type

 Exemples:
\end{flushleft}

   int iVarEntier 

   float fVarReel

   boolean bBooleen

   String sString

\begin{flushleft}
 Pour les nom d'instances de classes et de fonctions nous les commencerons par une minuscule pouis pour chaque nouveaux mots nous metterons une majuscule.

 Exemples:
\end{flushleft}

   Class exDeVariable
   int nomDeFonction()

\begin{flushleft}
 Pour la mise en page il ne faudra pas hésiter à faire des retours à la ligne pour ne pas avoir des lignes à rallonge.

 Exemples :

   - A la place de
\end{flushleft}

      if((this.attribut.methode() == classe.constante \&\& param.isSomething()) || (this.attribut2.methode() == classe2.constante \&\& param2.isSomething()))

\begin{flushleft}
    - Nous aurons
\end{flushleft}

      if((this.attribut.methode() == classe.constante

    	\&\& param.isSomething())

        || (this.attribut2.methode() == classe2.constante

        \&\& param2.isSomething()))
