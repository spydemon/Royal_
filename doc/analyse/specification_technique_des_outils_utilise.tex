\section{Etude technique des outils utilisé}

\subsection{Etude de ZXing}
La biblioteque ZXing \textit{(http://code.google.com/p/zxing/)} est un projet open-source multi-format de code-barres 1D/2D de traitement d'images mis en œuvre en Java. Ce projet met l'accent sur l'utilisation de la caméra intégrée sur les téléphones mobiles et de décoder les codes-barres sur l'appareil, sans communiquer avec un serveur.
De multiples formats pouvant être décodés ce qui nous permettera d'effectuer la capture de code barre sur notre application android.

Afin de faire appel à ZXing il est conseillé (car bien plus facile) de faire appel à l'application BareCodeScanner par le biais d'intent (ensembles de données qui peuvent être passé à un autre composant applicatif de la même application ou non). 
Mais cette solution ne semble pas être la plus optimal car il serait mieu d'éviter créé une dépendance à BareCodeScanner.
Pour ne pas avoir à faire appel à une autre application nous aurons à reprendre une partie de l'application BareCodeScanner que nous insererons directement dans notre application. 

