\section{Maquettes du programme}

\subsection{Maquettes de l'application android}

  \begin{figure}[htbp]
  \begin{center}
    \leavevmode
    \subfloat[Première page de l'application Royal\_Scanner]{%
		\label{}
		\includegraphics[height=7cm]{../Image_Royal_Scanner/First_Screen_Royal_Scanner.png}}
    \hspace{4cm}
    \subfloat[Menu de l'application Royal\_Scanner]{%
		\label{}
		\includegraphics[height=7cm]{../Image_Royal_Scanner/Menu_Royal_Scanner.png}}
  \end{center}
\end{figure}

Sur c'est deux image nous pouvons voir la page de demarage de royal ainsi que le menu proposant les differentes fonctionnalitées de l'application Royal\_Scanner.


\begin{figure}[htbp]
  \begin{center}
    \leavevmode
    \subfloat[Capture en cours]{%
		\label{}
		\includegraphics[height=7cm]{../Image_Royal_Scanner/Scan_En_Cours.png}}
    \hspace{4cm}
    \subfloat[Capture terminée]{%
		\label{}
		\includegraphics[height=7cm]{../Image_Royal_Scanner/Scan_Ok.png}}
  \end{center}
\end{figure}

Lorsqu'on lance une sequence de capture de code barre (simple ou multiple) un procedure basé sur la librarie Java ZXing qui permet la capture de code barre se lancera.
\newpage{}


\begin{figure}[htbp]
  \begin{center}
    \leavevmode
    \subfloat[Isbn correspondant valide (phase capture simple)]{%
		\label{}
		\includegraphics[height=7cm]{../Image_Royal_Scanner/Scan_Simple.png}}
    \hspace{1cm}
    \subfloat[Isbn correspondant non valide]{%
		\label{}
		\includegraphics[height=7cm]{../Image_Royal_Scanner/Erreur_test_isbn.png}}
    \hspace{1cm}
    \subfloat[Isbn correspondant valide (phase capture multiple]{%
		\label{}
		\includegraphics[height=7cm]{../Image_Royal_Scanner/Scan_Multiple.png}}
  \end{center}
\end{figure}

Après la procedure de capture l'application verifiera le code barre scanné.S'il est bon et que l'on est en capture simple l'application proposera la validation de l'album pour l'envoyer.
S'il est faut elle proposera un nouveau scan et enfin s'il est bon et que l(on est en capture multipple l'application proposera d'effectuer un nouveau scan ou d'envyer les ISBN des code barres capturés.

\begin{figure}[htbp]
  \begin{center}
    \leavevmode
    \subfloat[Configuration email]{%
		\label{}
		\includegraphics[height=7cm]{../Image_Royal_Scanner/Configue_Email.png}}
    \hspace{1cm}
    \subfloat[Synchronisation des albums déja present dans Royal\_]{%
		\label{}
		\includegraphics[height=7cm]{../Image_Royal_Scanner/Synchronisation_Album.png}}
    \hspace{1cm}
    \subfloat[Page à propos]{%
		\label{}
		\includegraphics[height=7cm]{../Image_Royal_Scanner/A_Propos.png}}
  \end{center}
\end{figure}

Lors de la configuration de l'adresse email nous aurons à entrer la nouvelle adresse email ,son mots de passe, on selectionnera son prototype (imap ou pop3) puis on ecrira le serveur utilisé.
Le prototype de la page de synchronisation n'est pas tres detailé car il s'agit d'une fonction de priorité 3 que l'on developpera plus tard dans le projet et que l'on definira la methode de transfere la plus performante une fois nos connaissance en developpement android plus grande. 
Cette page permettra sois la synchronisation par la recherche d'un email contenant tout les Isbn des album contenu dans royal soit la recuperation d'un fichier contenant c'est Isbn (fichier qui sera transmit soit par email soit directement par usb).
La page à propos permettera à l'utilisateur de trouver plus d'information sur Royal\_Scanner l'application.

\newpage{}

\subsection{Maquette de l'application PC}

  \begin{figure}[htbp]
  \begin{center}
    \leavevmode
    \subfloat[Application Royal\_ : fenêtre principale]{%
		\label{}
		\includegraphics[height=8cm]{../img/app_pc_maquette.png}}
  \end{center}
\end{figure}

Cette maquette présente une des vrais seul modification visible que va connaitre l'interface de base, c'est à dire le bouton "Importer" et le bouton "Synchroniser", le premier se chargeant de récupérer les mails envoyer depuis Royal\_Scanner et d'ajouter les albums en conséquence, et le second chargé de synchroniser les livres des deux applications (PC et Android) pour qu'elles contiennent les mêmes informations.

\begin{figure}[htbp]
  \begin{center}
    \leavevmode
    \subfloat[Configuration email]{%
		\label{}
		\includegraphics[height=5cm]{../img/preferenceMail.png}}
  \end{center}
\end{figure}

Dans l'optique de pouvoir récupérer les mails envoyer depuis l'application Android, l'utilisateur pourra, dans le panneau des préférences de l'application, renseigner les informations de sa boite e-mail à partir de laquelle l'import présenté ci-dessus sera effectué.

\newpage{}

\begin{figure}[htbp]
  \begin{center}
    \leavevmode
    \subfloat[Recherche des e-mails]{%
		\label{}
		\includegraphics[height=4cm]{../img/import_mail.png}}
    \hspace{1cm}
    \leavevmode
    \subfloat[Importation des lots d'albums]{%
		\label{}
		\includegraphics[height=3cm]{../img/ajout_import.png}}
  \end{center}
\end{figure}

La première fenêtre s'ouvrent après un clic sur le bouton "Importer" de l'interface principale, si l'adresse e-mail a bien été configuré, sinon l'utilisateur est inviter grâce à une boite de dialogue à entrer ces informations de boite de messagerie dans les préférences du logiciel. Si tout se déroule correctement, l'utilisateur visualise les courriels récupérés et peut choisir ceux qu'il désire traiter. Une fois son choix effectuer, le clic sur suivant déclenche le second volet (deuxième image), où l'utilisateur peut associer les livres récupérés dans chaque mail (correspondant à un lot) à une bilbiothèque, et lui affecter une date de retour.

