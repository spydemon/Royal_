\section{Affermissement des besoins exprimés} 

\subsection{Relative au client PC}
\subsubsection{Système d'ajout de d'édition de bande dessinées}

\paragraph{Ajout des informations de base}
Cette partie du programme est déjà implémentée par l'application Royal existante. 
Il suffira d'ajouter à celle-ci la possibilité d'enregistrer des informations en rapport avec les emprunts (bibliothèques et dates de retours). 

\paragraph{Gestion de l'image de couverture}
La gestion de l'image de couverture est déjà existante dans l'application Royal actuelle,
	mais devra subir des améliorations afin d'importer directement l'image depuis la base de données \emph{Google Books} et non pas depuis \emph{Google Images} comme c'est actuellement le cas,
	et qui ne fournit pas toujours des résultats fiables.

\paragraph{Création de liens entre les albums}	
Le partage des informations présentes dans les albums, comme les auteurs fait également partis des fonctionnalités déjà en place dans Royal. 
Cependant nous devrons l'adapter à la gestion des bibliothèques et des dates. 

\subsubsection{Système d'affichage des \emph{BD}s selon certains critères}

\paragraph{Affichage des \emph{BD}s}
Cette fonctionnalité est aussi implémentée avec les champs basiques de Royal.
À nous de l'adapter aux nouvelles données susceptibles d'être des bases de tri 
(à savoir les bibliothèques et dates de retours). 

\subsubsection{Système de recherche d'informations sur une \emph{BD} grâce au code barres}

\paragraph{Saisie et vérification du code barres}
La fonction vérifiera la bonne saisie d'un \emph{ISBN} et le mettra en forme dans une file de données qui sera utilisée par la suite. 
La fonction devra pouvoir traiter l'importation des \emph{ISBNs} depuis un courriel ou depuis une entrée manuelle de l'utilisateur. 

\paragraph{Recherche d'informations grâce aux codes barres}
Une fois la liste d'\emph{ISBNs} de nouveaux livres à insérer dans l'application de réalisée, 
	 celle-ci sera parcourue par une autre fonction qui se chargera de rechercher toutes les informations relatives à l'ouvrage sur \emph{Google Book}. 
La recherche s'effectuera livre après livre. 
Tous les livres en cours d'importation sont affichés dans une seule et même fenêtre. 
Les informations les concernant s'auto-complèterons en fonction de l'avancée de la recherche.
L'utilisateur pourra à tout moment les modifier, et confirmera lui même la fin de la saisie.C'est à ce moment que les ouvrages seront sauvegardés.

\subsubsection{Système d'importation des albums}

\paragraph{Importation via courriel} 
Le lien entre l'application Android et la PC s'effectuera via une boite courriel. 
Ainsi, l'application PC devra être en mesure de se connecter à la boite, effectuer une recherche de message, 
	le télécharger, et enfin le supprimer de la boite. 

\paragraph{Vérification du courriel}
À chaque lancement de l'application, une vérification sera effectuée afin de savoir si une adresse courriel à été configurée. 
Si ça n'est pas le cas, on invitera l'utilisateur à renseigner ces informations. 

\paragraph{Ajout automatique des albums importés}
Une fois un courriel téléchargé, l'application devra être en mesure d'y récupérer les courriels présent de les rendre « traitable » pour les autres fonctions du programme 
(à savoir, mettre les informations dans une liste de format compatible avec les fonctions de recherche présentées plus haut).  

\subsubsection{Système de gestion des lieux et dates d'emprunt}

\paragraph{Système d'alerte}
Au démarrage de l'application, une fonction de vérification des dates de retour sera effectuée par défaut. 
Si l'échéance avant la date buttoir de rendu d'un livre est plus basse qu'un seuil configurable, 
	alors une alerte sera émise à l'utilisateur. 
Celle-ci pourra être désactiveable par l'utilisateur si il n'éprouve pas le besoin d'être alerté. 

\section{Relative au client Android}

\subsection{Système de capture des codes barres}

\paragraph{Capture unique}
La fonction de capture unique se contentera d'appeler une fonction de la bibliothèque \emph{xzing} qui retournera un code barres récupéré via le capteur de l'appareil photo du téléphone.

\paragraph{Capture multiple}
La capture par lot englobe celle de capture unique.
La différence notable entre les deux fonctions et qu'après la capture d'un code barres valide, celle-ci donne la possibilité à l'utilisateur d'en scanner un autre à la volé,
	sans avoir à passer par le menu principal. 
Une fois que tous les codes seront scannés, l'utilisateur pourra indiquer la fin de l'enregistrement. 

\subsection{Vérification de la validité du code barre}
Une fois qu'un code barres a bien été scanné par l'application, 
	 une première vérification (locale) à lieu afin de savoir si il correspond effectivement à l'identification d'un livre, et non pas d'un objet quelconque. 
Une fois cette vérification passée avec succès, l'application recherche directement sur \emph{Google Book} le titre de l'ouvrage. Si celui-ci est trouvé, il s'affichera à l'utilisateur pour lui confirmer la bonne capture du livre. 
Si ce n'est pas le cas, l'utilisateur pourra rentrer le titre à la main. 

\subsection{Système d'enregistrement des albums déjà scannés}

\paragraph{Enregistrement des albums}
Pour lutter contre la redondance d'enregistrement des albums, une basse de données locale sera créée sur le téléphone (un simple fichier texte) qui contiendra tous les \emph{ISBNs} déjà scannés. 
De cette façon une alerte pourra être émise si l'utilisateur scann un livre qui l'aura déjà été. 

\paragraph{Synchronisation avec le client PC}
Comme le client Android n'aura pas forcément toutes les \emph{BDs} déjà enregistrées dans l'application PC, 
il faut ajouter une fonction de synchronisation pour que cette utilité prenne un sens.
Ainsi l'application Android pourra elle aussi se connecter sur la boite à courriel pour rechercher un éventuel message contenant la liste de tout les \emph{ISBNs} enregistrés sur l'application PC. 
Si c'est le cas, la liste locale sera effacée et remplacée par celle fraichement téléchargée.
