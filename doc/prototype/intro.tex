\documentclass{beamer}
\usepackage[utf8]{inputenc}
\usepackage[T1]{fontenc}
\usepackage{stmaryrd}
%\usepackage{cite}
\usetheme{moi2}
\title{Royal -- Le Prototype\\ T306AMAO}
\author[Jean, Kevin \& Steve]{Jean Meyblum \\ Kevin Hagner \\ Steve Benedick}
\institute{IUT Robert Schuman}
\date{\today}
\begin{document}

\begin{frame}
\titlepage
\end{frame} 

\begin{frame}{Les fonctionnalités}
\begin{columns}
\begin{column}[c]{5cm}
\underline{Sur l'application \emph{PC}}
\pause
\begin{itemize}
\pause \item Recherche d'informations grâce au code barres.
\pause \item Système d'importation des albums via courriel. 
\pause \item Système d'ajout automatique des albums importés.  
\end{itemize}
\end{column}

\begin{column}[c]{5cm}
\pause
\underline{Sur l'application \emph{Android}}
\begin{itemize}
\pause \item Système de capture des codes barres. 
\pause \item Système d'envois des albums via courriel.
\pause \item Vérification de la validité du code barre.
\end{itemize}

\end{column}
\end{columns}

\end{frame}

\begin{frame}{L'environnement technique}
\begin{columns}
\begin{column}[c]{5cm}
\pause
\underline{Sur l'application \emph{PC}}
\begin{itemize}
\pause \item Produit multi~-~plateforme.
\pause \item Traitement de la base de donnée : \emph{Hibernate}.
\end{itemize}
\end{column}

\pause 

\begin{column}[c]{5cm}
\underline{Sur l'application \emph{Android}} 
\begin{itemize}
\pause \item Android 2.2
\pause \item Base de donnée : \emph{SQLite}. 
\pause \item Utilisation de \emph{ZXing} pour traiter le code barres.
\end{itemize}
\end{column}
\end{columns}
\end{frame}

\begin{frame}
\textbf{Présentation…}
\pause
\textbf{Alors, convaincu ?}
\end{frame}

\begin{frame}{Ce qu'il reste à faire.}
\begin{columns}
\begin{column}[c]{5cm}
\underline{Sur l'application \emph{PC}}
\begin{itemize}
\pause \item Importation par un autre moyen.
\pause \item Gestion des lieux et dates d'emprunt.
\pause \item Synchronisation avec le client Android.
\end{itemize}
\end{column}

\pause 

\begin{column}[c]{5cm}
\underline{Sur l'application \emph{Android}}
\begin{itemize}
\pause \item Enregistrement des albums déjà scannés.
\pause \item Synchronisation avec le client PC.
\end{itemize}
\end{column}

\end{columns}
\end{frame}

\end{document}
