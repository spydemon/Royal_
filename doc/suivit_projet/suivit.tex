\documentclass[]{support-iutrs} % prof pour produire les notes
%\usepackage{slashbox}
%\usepackage{colortbl}
\setlength{\leftmargin}{0pt}

\setlength{\textheight}{670pt} 	% Hauteur de la zone de texte (25cm)
\infos{}{Jean Meyblum -- Kevin Hagner -- Steve Benedick}{T306AMAO}

\sujet{-- E31}
\titre{Suivi du projet}                          %optionnel
%\objectif{Objectifs du sujet}   %optionnel
%\notions{Notions importantes}   %optionnel


\begin{document}
	
\header

\section{Taches accomplies} 
\begin{tabular}{|l|r|r|r|r|r|}
\hline 
\textbf{Fonctionnalités} & \textbf{Jean} & \textbf{Kevin} & \textbf{Steve} & \textbf{Avcmnt} & \textbf{H rstt} \\
\hline
Saisie et vérification du code barres & 10 & & & 80\% & 2.00 \\
			 \hline
Recherche d'informations grâce au code - barres. & 15 & & & 98\% & 0.3 \\
\hline
Traitement en fonction du résultat de la recherche. & 10 & & & 50\% & 5 \\
				   \hline
Importation via courriel. & 20 & & & 85\% & 3 \\
					\hline
Configuration de l'adresse courriel & 8 & & & 100\% & 0 \\
					  \hline
Réception des courriers spécifiques & 20 & & & 100\% & 0 \\
		\hline
Ajout automatique des albums récupérés & 15 & & & 50\% & 7.5 \\
			\hline
Importation via un autre moyen & & & & 0\% & 0 \\
					\hline
Bibliothèque et date de retour & & 25 & & 20\% & 20 \\
		\hline
Système d'alerte pour les dates de retour & & 10 & & 10\% & 9.0 \\
		\hline
IHM \slash{} Design PC & 20 & & & 35\% & 13.0 \\
\hline 
Capture unique & & & 35 & 95 \% & 1.8 \\
			  \hline
Capture multiple & & & 2 & 95 \% & 0.1 \\
			  \hline
Configuration d'une adresse & & & 8 & 90\% & 0.8 \\
			\hline
Système d'envois par courriel & & & 21 & 90\% & 2.1 \\
		\hline
Système d'envois via un autre moyen & & & & 0\% & 0 \\
		\hline
Vérification de la validité du code barres & & & 3 & 100\% & 0.0 \\
		\hline
Affichage du titre d'un album scanné & & & 11 & 5 \% & 10.5 \\
				 \hline
Enregistrement des albums & & & & 0\% & 0.0 \\
						\hline
Mise en commun avec les albums du client PC & & & & 0\% & 0.0 \\
		  \hline
IHM \slash{} Design Android & & & 16 & 95\% & 0.8 \\
		 \hline
Optimisation du code & 10 & & 38 & 90\% & 4.8 \\
					 \hline
\textbf{Sous-total} & \textbf{128} & \textbf{35} & \textbf{134} & \textbf{59\%} & \textbf{123.1} \\
	\hline
\textbf{Organisation et documentation} & \textbf{Jean} & \textbf{Kevin} & \textbf{Steve} & \textbf{Avcmnt} & \textbf{H rstt} \\
	\hline
Documentation & & 30 & & 80\% & 6 \\
\hline
Gestion projet & & 40 & & 80\% & 8 \\
			  \hline
\textbf{Sous-total} & \textbf{0} & \textbf{70} & \textbf{0} & \textbf{80\%} & \textbf{14} \\
		\hline
\textbf{Total} & \textbf{128} & \textbf{105} & \textbf{134} & \textbf{60\%} & \textbf{137.1} \\
		\hline

\end{tabular} 

\section {Synthèse}

\begin{tabular}{|l|r|r|}
\hline
& Pourcentage effectué & Heures passées \\
		\hline
Fonctions de priorité 1 (Primaire) & 79\% & 203 \\
				 \hline
Fonctions de priorité 2 (Secondaire) & 29\% & 84 \\
				 \hline
Fonctions de priorité 3 (Facultative) & 10\% & 10 \\
				 \hline
\end{tabular} 

\section {La rétrospective}
\subsection{Ce qui a été fait}
La plupart les éléments de base ont étés effectués,
les applications \emph{PC} et \emph{Android} ont toutes les deux les bases nécessaires pour être utilisable. 

Ainsi nous avons réalisé, premièrement, sur l'application \emph{PC} : 
\begin{description}
\item [Recherches d'informations sur l'ouvrage -- ] Une recherche en fonction de l' \emph{ISBN} du livre pour importer directement les informations dans le programme comme le nom de l'auteur, la collection, le nombre de pages, etc. depuis internet. 
\item [Importation des \emph{ISBNs} -- ] L'importation des \emph{ISBNs} scannés depuis l'application \emph{Android} à aussi été implémentée. 
\end{description}

Et sur l'application \emph{Android} : 
\begin{description}
\item [Mise en place de la structure -- ] L'application de base, avec son design est fonctionnelle. 
\item [Capture des \emph{ISBN} -- ] Les \emph{ISBN} sont capturés par l'application. 
\item [Exportation des \emph{ISBN} -- ] Les \emph{ISBN} capturés sont correctement transmis à l'application \emph{PC}. 
\end{description}

\subsection{Ce qui a posé problème} 
Dans un premier temps, la classe \emph{JavaMail} n'est pas fiable sur \emph{Android} et sa compréhension à posé problème pour le développement du système d'envois des \emph{ISBNs} par email.
Il en de même pour \emph{zxSwing}, qui initialement mènent pas prévu pour être utilisé comme bibliothèque, mais comme application externe.
On a donc du réadapter le programme pour pouvoir l'importer dans notre application. 

An niveau de l'application \emph{PC}, la compréhension de \emph{Hibernate} pour la gestion de la base de données a (et est encore en fait) un problème assez complexe à gérer. 

\subsection{Ce qui aurait pu être amélioré}
Le format d'affichage des informations importés dans l'application du prototype aurait pu être un peu mieux travaillé. La détection du bug lors de l'importation des \emph{ISBNs} depuis le portable aurait aussi permis d'améliorer le prototype. 
\end{document}
