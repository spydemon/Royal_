\part{Définition des fonctionnalités}
Cette partie liste toutes les fonctionnalités attendues dans le programme. Celles-ci sont classées par priorité.

\begin{description}
\item [Priorité 1 :]
	Priorité la plus élevée. Il s'agit des premières fonctionnalités à implémenter.

\item [Priorité 2 :]
	Élément nécessaire. Il s'agit de l'implémentation réelle des fonctionnalités principales du programme mais nécessitant la mise en place de structures préalables (Les éléments de priorité 1). 

\item [Priorité 3 :]
	Éléments secondaires, ceux-ci serons effectués dans un second temps. Il s'agit d'améliorations du programme de base, mais pas de fonctionnalités nécessaires.  

\item [Priorité 4 :]
	Éléments optionnels. Ces fonctions ne sont pas demandées, mais rajoutent néanmoins une plus-value au programme. Elles serons implémentées uniquement si la vitesse d'avancée dans le projet nous le permet.
\end{description}


\section{Fonctionnalités de priorité 1}

%%%%% ÉLÉMENT DE PRIORITÉ 1

\subsection{Relatives au client PC}


\begin{description}
\item[Recherche des informations sur les livres]
	Le programme doit être capable de récupérer des informations sur les différents livres.
	Afin de récupérer ces informations, nous utiliserons l'API Google Book, car elle offre un grand nombre d'informations via l'ISBN du livre.

\item[Complétion automatique des champs]
	Après la récupération de ces informations, le programme auto-complètera tous les champs de caractéristiques de l'ouvrage dont la réponse a été trouvée.
	Si l'utilisateur souhaite néanmoins utiliser une autre information que celle fournie par Google Books, il pourra toujours modifier la valeur écrite automatiquement dans le champs.

\item[Ajouts des champs manquants]
	Royal ayant initialement été prévu pour la gestion des collections personnelles, il nous faudra ajouter des champs tels que le lieu de l'emprunt, la date de l'emprunt ainsi que la date de retour du livre afin de permettre la gestion des livres empruntés. 

\item[Connexion à une adresse Gmail]
	Le client qui gèrera la capture du code barre étant sous Androïd, la communication entre les deux programmes s'effectuant par mail, la plus simple des méthodes sera de passer par une boite Gmail.

\item[Récupération des mails]
	Cette fonction de lecture des mails par le client PC nécessitera une identification des mails ayant été envoyés par le client Androïd. 
	Cette identification se fera à l'aide d'une syntaxe spécifique sur le sujet du mail. Une fois les mails identifiés, le logiciel récupèrera les ISBN présents dans les messages puis les supprimera.  

\end{description}
\subsection{Relative au client Androïd}
\begin{description}

\item[Mise en place de l'interface graphique]
	Une interface graphique proposant un menu, donnant accès aux différentes fonctions de l'application du client Androïd, devra être créé. 
	Cette interface sera modifiée tout au long de la création du client Androïd afin de permettre l'accès au nouvelles fonctions.
	Elle se devra d'être élégante et ergonomique.

\item[Récupération et traitement du code barre]
	La librairie libre : ZXing \footnote{Site officiel de ZXing : http://code.google.com/p/zxing/} 
	permet le traitement de code-barres en Java. 
	C'est à dire qu'à l'aide d'une simple photo, elle est capable de repérer le code barre présent sur celle-ci et de retourner sa valeur.

\item[Envois du mail contenant l'ISBN via Gmail]
	Afin de faire parvenir le code ISBN au client PC, l'application Androïd devra, après le décodage du code-barres, envoyer l'ISBN correspondant par mail. 
	On utilisera tout d'abord le compte Gmail, car tout utilisateur Androïd en possède un. 
	Le mail aura une syntaxe spéciale dans son sujet afin d'être distingué, par le client PC, des autres mails.

\end{description}
%%%%% ÉLÉMENT DE PRIORITÉ 2

\section{Améliorations de priorité 2}

\subsection{Relative au client Androïd}

\begin{description}

\item[Vérification de la validité du code barre récupéré]
	L'application vérifiera que le code-barres récupéré est bien associé à un livre. 
	Si le code barre ne correspond pas à un livre (Si il ne commence ni par 978, ni par 979), ou si le code barre n'a pas été reconnu sur la photo, l'utilisateur en sera averti via un message d'erreur.

\item[Connexion à l'API Google Books depuis le client Androïd]
	Cette connexion a pour but la permission de confirmer, depuis le client Androïd, que celui-ci est répertorié dans la base de données Google. 
	Si jamais il n'y est pas, l'utilisateur pourra directement écrire le nom du livre depuis son téléphone, et ainsi, n'aura pas besoin de reconnaitre l'ouvrage ultérieurement par son ISBN.

\end{description}

\section{Améliorations de priorité 3}

\subsection{Relatives au client PC}

\begin{description}
\item[Connexion à une autre adresse mail]
	Cette fonction aura pour but de permettre à l'utilisateur de choisir entre l'adresse par défaut (l'adresse Gmail liée au compte Androïd) et une autre adresse que l'utilisateur choisirait lui-même. 
	Pour permettre de choisir d'autres adresses mails, le logiciel devra gérer plusieurs types de messagerie.
	Il faudra donc être capable de régler « manuellement » les paramètres de connexion tels que les serveurs IMAP / POP et SMTP à interroger.

\item[Stocker les lieux d'emprunts dans une base de données]
	Afin d'éviter à l'utilisateur de devoir écrire pour chacun des livres empruntés, le non de la bibliothèque dans laquelle l'emprunt a été effectué, une base de données stockant les lieux d'emprunt sera créé. 
	L'utilisateur n'auras plus qu'à sélectionner son lieu d'emprunt dans une liste. 
	Si le lieu d'emprunt n'existe pas dans la base de données, l'utilisateur pourra le rajouter.

\item[Gestion des ISBN indisponible sur Google Books]
	Pour les ouvrages dont l'ISBN n'est associé à aucune information sur Google Books, le logiciel affichera un message informant l'utilisateur que l'ajout automatique est impossible pour ce livre, afin que l'utilisateur puisse les entrer manuellement.   

\item[Ajout d'un système de rappel de date de retour]
	Le logiciel devra émettre des alertes à l'utilisateur quand la date de fin d'emprunt d'un livre approche. 
	Cette fonction devra pouvoir être désactivée par l'utilisateur s'il ne l'estime pas nécessaire.

\item[Conversion d'un ISBN 13 ou code barre]
	Un champs sera ajouté afin que l'utilisateur puisse entrer le code barre de son livre ou l'ISBN à 13 chiffres dans les informations sur une bd. Il sera converti en ISBN à 10 chiffres. C'est cette information qui sera sauvegardée dans la base de données du logiciel et utilisé.

\end{description}
\subsection{Relatives au client Androïd}
\begin{description}
 
\item[Envoie du mail contenant les ISBNs via une autre adresse mail]
	Cette fonction aura pour but de permettre à l'utilisateur de choisir entre l'adresse par défaut (l'adresse Gmail liée au compte Androïd) et une autre,
	qu'il aura choisi pour l'envoie des ISBNs au client PC. 

\end{description}

\section{Améliorations de priorité 4}
\subsection{Relatives au client PC}
\begin{description}

\item[Tri des bandes dessinées]
	L'affichage des bandes dessinées pourra être géré en fonction de la série, de l'auteur, de la date de parution ou de la date de lecture. 
	L'affichage en fonction de l'auteur ou de la collection est déjà une fonctionnalité présente sur l'application Royal mais elle est limitée et propose aussi un affichage en fonction du type alors que le type de l'ouvrage ne fait pas partie des informations que l'on peut saisir. 
	Cette révision de l'affichage aura pour principal but de faciliter la recherche de l'utilisateur parmi ses bandes dessinées.

\item[Amélioration de la recherche d'images]
	La fonction de recherche d'images implantée dans Royal ne propose pas toujours une image adaptée à notre album. C'est pourquoi nous allons attribuer automatiquement l'image de la couverture de la bande dessinée en fonction de celle présente sur Google Book. 
	Cette image pourra toujours être changée manuellement par une image personnelle comme on pouvait déjà le faire sur Royal.

\item[Synchronisation des ISBNs présentes dans la base de données]
	Afin d'améliorer la fonction du client Androïd vérifiant les ISBNs déjà existants, 
	une fonction de synchronisation sera implémentée afin de lui faire parvenir les ISBNs des bandes dessinées déjà présentes dans le client PC susceptibles de manquer sur le client Android. 
	(Si le client lourd est utilisé depuis avant la création du client Androïd, ou si l'utilisateur change de téléphone par exemple). 
	Cette synchronisation sera faite par email ou usb (peut être par bluetooth ou wifi).

\item[Réception de l'ISBN par un autre moyen]
	Cette fonction permet à l'utilisateur de recevoir les codes ISBNs par un autre moyen que l'envoi d'un mail. 
	(Solutions à étudier: usb, bluetooth, serveur, etc. )

\end{description}
\subsection{Relatives au client Androïd}
\begin{description}

\item[Gestion des ISBNs des bandes dessinées déjà lues]
	Tous les ISBNs des bandes dessinées scannées seront conservés dans un fichier sous l'application Androïd afin de garder un historique des bandes dessinées déjà lues.
	Si l'ISBN a déjà été scanné précédemment, un message d'information sera envoyé au client. 
	Toutefois il lui sera demandé s'il veut quand même renvoyer le code ISBN au cas ou le livre aurait été supprimé sur le client PC. 

\item[Synchronisation de l'historique des ISBN des livres déja lus]
	Une synchronisation des ISBN présents sur le client PC, avec le client Androïd sera possible. 
	L'historique présent dans l'application Androïd sera effacé et remplacé par ceux importé depuis le client PC.

\item[Envois de l'ISBN par un autre moyen]
	Cette fonction permet d'envoyer les ISBNs des livres scannés au client PC par un autre moyen que l'envoi d'un mail. 
	(Solution à étudier: usb, bluetooth, serveur, etc.)

\end{description}
