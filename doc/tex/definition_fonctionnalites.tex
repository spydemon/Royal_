\part{Définition des fonctionnalités}
Cette partie liste toutes les fonctionnalités attendues dans le programme. Celle - ci sont classées par priorités.

\begin{description}
\item [Priorité 1 :]
	Priorité la plus élevée. Il s'agit des premières fonctionnalités à implémenter.

\item [Priorité 2 :]
	Élément nécessaire. Il s'agit de l'implémentation réelle des fonctionnalités principales du programme mais nécessitant la mise en place de structures préalables (Les éléments de priorité 1). 

\item [Priorité 3 :]
	Éléments secondaires, ceux - ci serons effectués dans un second temps. Il s'agit d'améliorations au programme de base, mais pas de fonctionnalités nécessaires.  

\item [Priorité 4 :]
	Éléments de « luxe ». Ces fonctions ne sont pas demandés mais rajoutent néenmoins une plu value au programme. Elles serons implémentées uniquement si la vitesse d'avancée dans le projet nous le permet.
\end{description}


\section{Amélioration du client lourd}

%%%%% ÉLÉMENT DE PRIORITÉ 1
\subsection[Vérification du code bar]{Vérification du code bar — \emph{Priorité 1}}
Après l'entrée du code barre par l'utilisateur le programme devra executer une verification de ce code barre afin de verifier s'il s'agit bien d'un code par correspondant à un livre ou un autre code barre. Pour effectuer cette verification le programme devra verifier si le code barre commence par 978 ou 979 et s'il est bien composé de 13 chiffres. Si c'est conditions sont bien vérifié le code barre sera validé. 

%%%%% ÉLÉMENT DE PRIORITÉ 2

%%%%% ÉLÉMENT DE PRIORITÉ 3

%%%%% ÉLÉMENT DE PRIORITÉ 4

\section{Création du client Androïd}

%%%%% ÉLÉMENT DE PRIORITÉ 1

%%%%% ÉLÉMENT DE PRIORITÉ 2

%%%%% ÉLÉMENT DE PRIORITÉ 3

%%%%% ÉLÉMENT DE PRIORITÉ 4

