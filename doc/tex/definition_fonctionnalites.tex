\part{Définition des fonctionnalités}
Cette partie liste toutes les fonctionnalités attendues dans le programme. Celle - ci sont classées par priorités.

\begin{description}
\item [Priorité 1 :]
	Priorité la plus élevée. Il s'agit des premières fonctionnalités à implémenter.

\item [Priorité 2 :]
	Élément nécessaire. Il s'agit de l'implémentation réelle des fonctionnalités principales du programme mais nécessitant la mise en place de structures préalables (Les éléments de priorité 1). 

\item [Priorité 3 :]
	Éléments secondaires, ceux - ci serons effectués dans un second temps. Il s'agit d'améliorations au programme de base, mais pas de fonctionnalités nécessaires.  

\item [Priorité 4 :]
	Éléments de « luxe ». Ces fonctions ne sont pas demandés mais rajoutent néenmoins une plu value au programme. Elles serons implémentées uniquement si la vitesse d'avancée dans le projet nous le permet.
\end{description}


\section{Amélioration du client lourd}

%%%%% ÉLÉMENT DE PRIORITÉ 1
\subsection[Vérification du code bar]{Vérification du code bar — \emph{Priorité 1}}
Après l'entrée du code barre par l'utilisateur le programme devra executer une verification de ce code barre afin de verifier s'il s'agit bien d'un code par correspondant à un livre ou un autre code barre. Pour effectuer cette verification le programme devra verifier si le code barre commence par 978 ou 979 et s'il est bien composé de 13 chiffres. Si c'est conditions sont bien vérifié le code barre sera validé. 

\subsection[Recherche sur l'API Google book]{Recherche sur l'API Google book — \emph{Priorité 1}}
Le programme doit etre capable de recuperer des information sur les differents livres. Afin de recuperer ces informations nous utiliserons l'API Google book car elle offre un grand nombre d'information grace à L'ISBN du livre (ISBN = code barre sans les trois premiers chiffres). Apres la récupération de ces information le programme les stockeras dans un objet information comptenant les differentes information recolté néccésaire à Royal.  

\subsection[Complétion automatique des champs]{Complétion automatique des champs — \emph{Priorité 1}}
Une fonction de complétion auomatique des champs de caractéristiques de la bande déssiné devra ètre créé. Cette fonction remplira automatiquement les differents champs grace a l'objet information créé par la fonction de recherche sur l'API Google book.

\subsection[Ajouts des champs manquants]{Ajouts des champs manquants — \emph{Priorité 1}}
Royal ayant initialement été prevu pour la gestion des collections personnels il nous faudra ajouter des champs tels que le lieu de l'emprunts, la date de l'emprunts ainsi que la date de retours du livre afin de permettre la gestion de livres empruntés. 

\subsection[Connexion à une adresse Gmail]{Connexion à une adresse Gmail — \emph{Priorité 1}} 
Sachant que les differents codes barres scanner a l'aide du smartphone sous android seront envoyer via une adresse Gmail, il nous faudra gerer la connexion à cette adresse Gmail afin de récupérer les codes barres.

\subsection[Récupération des mail]{Récupération des mail — \emph{Priorité 1}} 
Cette fonction de lecture des mail par le client lourd nécessitera une identification des mail ayant été envoyer par le client android. Cette identification se fera à l'aide d'une syntaxe spécifique du sujet du mail. Une fois le mail identifié le logiciel recuperrera l'ISBN et suppimera le message. 

%%%%% ÉLÉMENT DE PRIORITÉ 2


%%%%% ÉLÉMENT DE PRIORITÉ 3

\subsection[Connexion à une autre adresse mail]{Connexion à une autre adresse mail — \emph{Priorité 3}}
Cette fonction aura pour but de permettre à l'utilisateur de choisir entre l'adresse par defaut (l'adresse Gmail liée au compte android) et une autres adresse qu'il choisirait lui-même. Pour permettre de choisir d'autre adresse mail le logiciel devra gerer plusieurs types de comptes de messagerie tels que ceux utilisant des compte POP3 ou encors IMAP.

\subsection[Stocker les lieu d'emprunts dans une Base De Donnée]{Stocker les lieu d'emprunts dans une Base De Donnée — \emph{Priorité 3}}
Afin d'éviter à l'utilisateur de rajouter le nom du lieu d'emprunts pour chacun des livres empruntés, une base de donnnée stockant des lieux d'emprunts sera créé. L'utilisateur n'oras plus qu'à selectionner sont lieu d'emprunts dans une liste. Si le lieu d'emprunt n'existe pas dans la base de donnée l'utilisateur poura le rajouter.

\subsection[Gestion des ISBN indisponible sur Google Docs]{Gestion des ISBN indisponible sur Google Docs — \emph{Priorité 3}}
Pour les ouvrage dont l'ISBN n'est associé à aucune information sur Google Docs le logiciel affichera un message informant l'utilisateur que l'ajouts automatique est impossible pour ce livre afin que l'utilisateur puisse les entré manuellement.   

\subsection[Ajout d'un système de rappel de date de retour]{Ajout d'un système de rappel de date de retour — \emph{Priorité 3}}
Le logiciel devra émettre des alertes à l'utilisateur quand la date de fin d'emprunt d'un livre approche. Cette fonction devra pouvoir etre désactivé par l'utilisateur s'il ne desir pas l'utilisé afin d'évité de le géner.

%%%%% ÉLÉMENT DE PRIORITÉ 4

\subsection[Tri des bande dessinée]{Tri des bande dessinée — \emph{Priorité 4}}
L'affichage des bande dessinée pourra etre gerer en fonction de la serie, de l'auteur, de la date de parution ou de la date de lecture. Un affichage en fonction de l'auteur ou de la collection est deja presents sur l'apllication Royal mais il est limité et propose aussi un affichage en fonction du type alors que le type de l'ouvrage ne fais pas partie des information que l'on peut saisir. Cette revision de l'affichage aura pour principal but de facilité la recherche de l'utilisateur parmi ses bande dessinées.

\subsection[Amélioration de la recherche d'image]{Amélioration de la recherche d'image — \emph{Priorité 4}}
La fonction de recherche d'image proposé par royal ne propose pas toujours une image adaptée à notre album c'est pourquoi nous allons attribué automatiquement l'image de la couverture de la bande dessinée en fonction de celle présente sur Google Book. Cette image pourra toujours etre changé manuellement par une image personelle comme on pouvait le faire avec royal.
%Sur mon royal il n'enregistre pas les couverture :s

\subsection[Syncronisation des bandes dessinées déja présentes dans la bases de donnée]{Syncronisation des bandes dessinées déja présentes dans la bases de donnée — \emph{Priorité 4}}
Afin d'améliorer la fonction du client android verifiant les ISBN deja existant, une fonction de syncronisation avec le client android sera crée afin de lui faire parvenir les SBN des bandes dessinée deja présentes dans le client lourd. Cette syncronisation sera faite par email, bluetooth ou usb.

\section{Création du client Androïd}

%%%%% ÉLÉMENT DE PRIORITÉ 1

%%%%% ÉLÉMENT DE PRIORITÉ 2

%%%%% ÉLÉMENT DE PRIORITÉ 3

%%%%% ÉLÉMENT DE PRIORITÉ 4

