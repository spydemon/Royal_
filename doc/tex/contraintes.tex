\part{Définition des contraintes}


\section{Contraintes de temps}
Pour notre projet, nous avons un temps imparti avec au moins trois étapes, donc au moins trois livrables. Ces remises ne nous permettent aucun retard donc chaque étape du projet devra être terminée dans les temps impartis.
Les autres contraintes de temps que nous pourrons rencontrer lors de ce projet sont les heures que l'on consacrera à nos études par ailleurs (par exemple pendant les partiels). 

\section{Contraintes de qualités}

\subsection{Charte ergonomique}
Notre projet étant l'amélioration d'un programme déjà existant, nous devrons respecter la charte graphique déjà existante afin de concorder avec le reste du programme. 
De même, pour l'application Android, nous devrons nous adapter à une utilisation tactile de l'application. 

\subsection{Norme d'écriture}
Royal est un projet libre. 
Nous nous devons de redistribuer le code source que l'on créera.
Pour se faire, il faudra mettre en place des normes d'écritures afin de faciliter la compréhension de notre code par d'autres personnes susceptibles de reprendre le projet. 
Nous devrons également écrire les commentaires dans notre programme en anglais, comme c'est le cas dans Royal à présent.

\section{Contraintes matérielles et technologiques}

\subsection{Adaptation aux outils utilisés dans Royal}
Pour notre projet, nous devons nous adapter aux outils déjà utilisés. Une bonne partie de notre travail consistera à comprendre le fonctionnement de ces outils.

\subsection{Compatibilité du client PC}
Le client PC devra pouvoir s'installer sous \emph{Linux} et si possible sur \emph{Windows} et \emph{MacOS}. 
Pour permettre cette comptabilité, nous fournirons les librairies adaptées aux différents systèmes d'exploitation avec notre programme. 

\subsection{Compatibilité du client mobile}
L'application mobile sera compatible avec les téléphones Android à partir de la version 2.2. 
Nous devrons aussi faire attention à ce que les différentes librairies utilisées soient déjà installées sur le téléphone ou, dans le cas contraire, les installer automatiquement.

\section{Contraintes de réalisation}

\subsection{Designation des acteurs}
Pour ce projet, notre maître d'ouvrage est \emph{Ogier Maitre}, et l'équipe de maîtres d'œuvre est composée de \emph{Kévin Hagner}, le chef de projet, \emph{Jean Meyblum} et \emph{Steve Benedick}.
Afin de faciliter la communication entre notre tuteur et nous, nous avons choisi d'organiser des réunions une à deux fois par semaine. 
En plus de cela, le développement pourra être suivi sur notre dépôt Git.
\footnote{Page Github de notre depot Git: https://github.com/spydemon/Royal\_} 
En ce qui concerne la communication entre les membres de notre trinôme, nous avons créé un canal IRC \footnote{ canal \#royal\_ sur le serveur irc.rezosup.org} afin de faciliter la discution à plusieurs. 
De même, nous pourrons travailler ensemble soit à l'IUT soit chez l'un des membres si besoin.

\subsection{Cycle de vie}
Nous avons choisi d'adapter notre projet au cycle de vie \emph{ICAR} pour deux principales raisons : 

La première, c'est que la structure même du T3 nous impose son utilisation :
en effet, nous devons livrer un cahier des charges, puis une analyse fonctionnelle, suivie d'un prototype, et enfin de la version finale. 
Un cycle en V, par exemple aurait été assez difficile à tenir avec ce mode de fonctionnement vu que toutes les étapes « se mélangent » un peu. 

La seconde vient justement de la structure claire qu'elle exige. 
En effet, tout le projet est hiérarchisé (on doit, par exemple d'abord finir le cahier des charges avant de s'attaquer à l'analyse fonctionnelle). 
De cette façon, l'effet didactique du projet est amplifié, car nous apprenons d'abord à rédiger correctement, puis à coder. 
De plus, le contrôle est plus efficace ce qui permet de nous corriger assez rapidement en cas d'erreur.
