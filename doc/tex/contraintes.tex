\part{Définition des contraintes}


\section{Contraintes de temps}
Pour notre projet nous avons un temps imparti avec au moins trois étapes donc au moins trois livrables. Ces remises ne nous permettent aucuns retard donc chaque étape du projet devra être terminée dans les temps impartis.
Les autres contraintes de temps que nous pourront rencontrer lors de ce projet sont les heures que l'on devra consacrer à nos études de façon extérieure au projet (par exemple pendant les partiels). 

\section{Contraintes de qualités}

\subsection{Charte ergonomique}
Notre projet étant l'amélioration d'un programme déjà existant nous aurons à respecter la charte graphique déjà existante afin de concorder avec le reste du programme. 
De même, pour l'application Androïd, nous aurons à nous adapter à une utilisation tactile de l'application. 

\subsection{Norme d'écriture}
Royal est un projet open - source. 
Nous nous devons de redistribuer le code source qu'on créera.
Pour ce fait, des normes d'écritures doivent êtres mise en place afin de faciliter la compréhension de notre code par d'autres personnes susceptible de reprendre le projet. 
Nous devrons également écrire les commentaires dans notre programme en anglais, comme c'est le cas dans Royal à présent.

\section{Contraintes matérielles et technologiques}

\subsection{Adaptation aux outils utilisés dans Royal}
Pour notre projet nous devons nous adapter au outils déjà utilisés. Une bonne partie de notre travail consistera à comprendre le fonctionnement de ces outils.

\subsection{Compatibilité du client lourd}
Le client lourd devra pouvoir s'installer sous Linux et si possible sur Windows et MacOS. 
Pour permettre cette comptabilité nous devrons veiller à fournir les différentes librairies adaptées aux différents systèmes d'exploitation avec notre programme. 

\subsection{Compatibilité du client mobile}
L'application mobile devra être compatible avec les téléphones Androïd à partir de la version 2.2. 
Nous devrons aussi faire attention à ce que les différentes librairies utilisées soient déjà installées sur le téléphone, ou dans le cas contraire les installer automatiquement.

\section{Contraintes de realisation}

\subsection{Designation des acteurs}
Pour ce projet notre maître d'ouvrage est Maitre Ogier, pour ce qui est des maître d'oeuvre notre équipe est composée de Hagner Kévin le chef de projet, Meyblum Jean ainsi que Benedick Steve.
Afin pour faciliter la communication entre notre tuteur et nous, nous avons choisi d'organiser des réunions une à deux fois par semaine. De même Maitre Ogier pourra suivre le développement de notre projet sur notre dépot Git.
\footnote{Page Github de notre depot Git: https://github.com/spydemon/Royal\_} 
Pour ce qui est de la communication entre les membres de notre trinôme nous avons crée un canal irc \footnote{ canal \#royal\_ sur le serveur irc.rezosup.org} afin de pouvoir faciliter la discution à plusieurs. 
De même nous pourrons travailler ensemble soit a l'IUT soit chez l'un des membres si besoin.

\subsection{Cycle de vie}
Afin de développer parallèlement différents modules une fois la phase notre cahier des charge validé nous avons choisi un cycle de vie en W. 
En effet comme dans ce modèle l'intégration est déja étudiée lors de la conception globale des fonctionnalités nous pouvons concevoir, développer et tester chaque fonctionnalitée séparément. 
De même ce cycle de vie nous permet de développer notre projet en fonction des priorités de nos tâches ce qui nous permet de garantir au minimum le rendu d'un projet répondant aux demandes de base.
