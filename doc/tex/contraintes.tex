\part{Définition des contraintes}


\section{Contraintes de temps}
Pour notre projet nous avons un temps imparti avec au moin trois étapes donc au moin trois livrables. Ces remises ne nous permettent aucuns retard donc chaque étapes du projet devra etre terminé avant l'arrivé de l'écheance limite. Les autres contraintes de temps que nous pouront rencontré lors de ce projets sont les heures que l'on devra consacré à la révision de nos cours qui nous empecherons de travailler sur notre projet quelque fois.  

\section{Contraintes de qualités}

\subsection{Charte ergonomique}
Notre projet etant l'amelioration d'un programme déja existant nous aurons a respecter sa charte graphique afin de concorder avec le reste du programme. De même, pour l'application android, nous aurons à nous addapter a une utilisation tactile de l'application. Nous devron veillez à ce que l'utilisateur puisse simplement acceder au élément sans toucher à celui à coté.  

\subsection{Norme d'écriture}
Royal est un projet open-source c'est pourquoi ne devont respecter certaine norme d'écriture afin de faciliter la comprehention de notre code par d'autre personne. Pour ce qui est des commentaire de notre code pour concorder avec le reste du programme il faut que nous les ecrivions en anglais.

\section{Contraintes matérielles et technologiques}

\subsection{Adaptation au outils utilisé dans Royal}
Pour notre projet nous devons nous adapter au outils déja utilisé, une bonne partie de notre travail consistera a conprendre le fonctionnement de ces outils.

\subsection{Compatibilité du client lourd}
Le client lourd devra pouvoir etre installer obligatoirement sur linux et si possible sur windows et macOS. Pour permettre cette comptabilité nous devrons veillez a fournir les differentes library adapté au differents systeme d'exploitation au utilisateurs. 

\subsection{Compatibilité du client mobile}
L'application mobile devra etre compatible avec les téléphone android à partir de la version 2.2. Nous aurons aussi à faire attention a ce que les differente library que nous utilisons soit disponible sur tout les téléphone android d'une 2.2 ou de les faire installer avec l'application.

