\part*{Introduction}
Notre projet a pour but la gestion informatique des emprunts et achats de bandes dessinées (BD) réalisée par un particulier.
Pour cela, nous aurrons à mettre en place un logiciel capable: 
\begin{itemize}
\item Gerer la saisie et l'enregistrement de bandes dessinées.
\item Automatier l'ajout d'une BD grâce à son code barre qui permettra la recherche et le remplissage automatique de l'ensemble des informations de l'ouvrage.
\item Ajouter d'un système de bibliothèque, de durée d'emprunt, de date d'emprunt, tout cela pour pouvoir notifier à l'utilisateur s'il doit penser à rendre certains de ses ouvrages et d'où proviennent ses ouvrages.
\item La synchronisation avec l'application Android afin de récupérer les codes barres.
\end{itemize}

Une application mobile compatible avec les Smartphone Android \textit{(systeme d'éxploitation mobile développé par google utilisé par un grand nombres de constructeur)} sera aussi développée afin de permettre:
\begin{itemize}
\item La capture des code-barres d'un ensemble de BD.
\item L'envoie des code-barres à l'application qui grâce à la synchronisation sera capable d'ajouter automatiquement un lot de BD dans sa base de donnée.
\end{itemize}

Afin de gerer la saisie et l'enregistrement des bandes déssinées nous allons utiliser d'une application déja existante, 
\emph{Royal} \footnote{Site officiel du projet Royal : www.royal-project.org}. 
L'utilisation d'un logiciel déja éxistant nous permettera de mieux nous consentrer sur la gestion des emprunts ainsi que la capture de code barres via l'application mobile.

Le projet est à but personnel, il n'a comme unique objectif que de servir d'aide mémoire à un utilisateur.
Il ne s'agit pas, par exemple d'un système de gestion de bibliothèque ou de stocks des bd restantes dans une librairie.
Aucune relation ne sera possible entre différents utilisateurs, ni de mises en commun des œuvres lues
(statistiques des livres les plus lus par la « communauté », notation, commentaires…).
