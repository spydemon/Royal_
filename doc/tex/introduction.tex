\part*{Introduction}
Notre projet a pour but la gestion informatique des emprunts et achats de bandes dessinées (bd) réalisé par un particulier.
Pour cela nous allons partir d'une application existante, \emph{Royal}
\footnote{Site officiel du projet Royal : www.royal-project.org}, qui permet la saisie et l'enregistrement de bandes dessinées, et l'améliorer.
En effet, Royal ne permet qu'une saisie manuelle de chaque ouvrage, et ne gère ni les bibliothèques d'emprunt ni les dates d'emprunt et de retour. Ainsi nous allons permettre à un utilisateur de simplifier sa gestions de bd, d'une part en améliorant l'application Royal et d'une autre part en développant une application mobile pour Android.

L'amélioration consistera en :
\begin{itemize}
\item L'automatisation de l'ajout d'une bd grâce à son code barre qui permettra la recherche et le remplissage automatique de l'ensemble des informations de l'ouvrage.
\item L'ajout d'un système de bibliothèque, de durée d'emprunt, de date d'emprunt, tout cela pour pouvoir notifier à l'utilisateur s'il doit penser à rendre certains de ses ouvrages, d'où proviennent ses ouvrages et combien de temps il lui reste pour les lires.
\item La synchronisation avec l'application Android.
\end{itemize}

L'application Android doit permettre :
\begin{itemize}
\item La capture des codes barres d'un ensemble de bd.
\item L'envoie des codes barres à l'application améliorée de Royal qui grâce à la synchronisation sera capable d'ajouter automatiquement un lot de bd.
\end{itemize}

Le projet est à but personnel, il n'a comme unique objectif que de servir d'aide mémoire à un utilisateur.
Il ne s'agit par exemple pas d'un système de gestion de bibliothèque ou de stocks des bd restantes dans une librairie.
Aucunes relation ne sera possible entre différents utilisateurs, ni de mises en commun des œuvres lues
(statistiques des livres les plus lus par la « communauté », notation, commentaires…).
