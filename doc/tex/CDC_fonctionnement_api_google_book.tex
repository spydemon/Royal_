Recherche avec L'API Google Book

Cette fonctionnalité principale permet de récupérer les informations sur une
bande dessinée grâce à son code barre. En effet L'API Google Book possède une
bibliothèque JAVA lui étant dédiée qui permet de construire un objet "livre"
grâce au code ISBN (déchiffrable à partir du code barre) d'une bande dessinée
et ainsi de trouver un ensemble d'information exploitable par la bédéthèque.
Cette fonctionnalité s'intègre dans l'otpique de l'automatisation des ajouts de
bandes dessinées sur le client lourd.

Les informations retournées par gBook sont :

  - le titre
  - le sous-titre
  - la description
  - la ou les catégories
  - le ou les auteurs
  - l'éditeur
  - la date de parution
  - le prix
  - les liens vers les images de couvertures

Fonctionnement de L'API : 