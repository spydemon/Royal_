\part{Planning prévisionnel}

\section{Etapes du projets}

Notre projets comporte 4 étapes principales. Chacune de ces étapes aura une date de remise définie des le debut du projet. 

\paragraph{La rédaction du cahier des charges:}
Dans cette étape nous avons eu principalement besoin de rediger la demande du client. 
Nous avons aussi eu à étudier le fonctionnement du logiciel Royal que l'on ameliore afin de définire les contraintes liées à l'amélioration de ce logiciels et nous permettre de definire la faisabilité des differentes fonctions. 
Un prototype du projet android à aussi du être concu afin de nous permettre de décrire une utilisation type.

\paragraph{La rédaction du Dossier d'analyse:}
Dans cette étape nous aurons à approfondire l'analyse ainsi que la concetion des differentes fonctions ainsi que le fonctinnement du logiciel Royal. 
Nous définirons aussi les différents outils utilisé afin de développer le projet tels que eclipse pour le developpement JAVA ou encors PowerAMC pour la modification de la Base de Donnée.
Nous effectuerons aussi les modification du models de la base de donnée ainsi que les premiers models de communication.

\paragraph{La réalisation d'un premier prototype:}
Afin de rendre un prototype, le 5 Decembre 2011, répondant au principal demande de l'utilisateur nous développerons les fonctions de prioritées 1 et 2.
Pour permettre la création de ces fonctions nous aurons à modifier la base de donnée en suivant le models modifié dans l'etape précédentes.  

\paragraph{La finalisation du projet:}
Cette étape sera la dernière de notre projet et devra être fini avant le 19 Janvier 2012 (date de livraison de notre projet).
Pour cette étape nous développerons les fonctions de priorités inferieur afin de répondre à des besoin moin important. 
Nous essayerons de rendre les de fonctionnalité de priorité secondaire décrite dans le cahier des charge ainsi que certaines fonctionnalité de confort(en fonction de la vitesse de développement du projet).


