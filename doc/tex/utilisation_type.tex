\section{Utilisation type}
Dans cette partie sera évoqué un exemple d'utilisation type du programme.
Les actions de l'utilisateur serons décrites, et les processus techniques (en italique) que l'action engendre également. 

\subsection{Prélude}
Nous sommes mercredi soir, 17h30, le cour viens de se terminer. 
Je marche tranquillement vers l'arrêt de tram pour rentrer chez moi.
Le ciel est bleu, il fait chaud, mais une douce brise venant du nord se fait sentir. 
Je monte dans le tram, et m'installe tranquillement, comme à mon habitude sur une chaise au fond de celui - ci. 
Le tram part peut de temps après, et commence sont habituelle trajet…

Quand soudain, un épouvantable drame arriva ! 
Un camion rempli de machine à laver perd toute sa cargaison en plein milieu du carrefour à Étoile Bourse, évidemment, les voies du tram sont inutilisable. 
La foule était en panique la plus totale, et criait à la mort tel qu'on eu cru que le ciel nous tombait sur la tête. 
Moi, dans tout mon désarroi, et toute ma tristesse pensait à ces précieuses minutes de perdu sur mon T3 à attendre que la voie soit de nouveau praticable…

C'est alors que pendant un long soupir, mon regard se posa sur le toit de la bibliothèque Andrée Malraux. 
Instantanément, j'ai repensé à cette série de bd que mon ami Steve m'avait conseillé quelques heures plus tôt : XIII
\footnote{Informations sur la BD XIII : http://fr.wikipedia.org/wiki/XIII\_(bande\_dessinée)}. 
Je décide donc d'abandonner le tram bloqué, et d'oublier le T3 qui m'attendait à la maison, pour me rendre dans la bibliothèque, et voir ce qu'il en est. 

Quelle ne fut pas mon impression quand une fois les portes franchies, je découvris l'immensité du bâtiment !
Un grand hall s'offrait à moi, dans le quel était présent un grand snack et, en arrière - plan, des djihadistes s'informant des nouvelles de leurs pays en regardant Al - jazeera. 
Je décide de partir à la conquête de cet amas de béton et de métal, inspirant plus à un bunker anti - atomique qu'à une bibliothèque. 
La recherche fut longue et périlleuse, mais ma lutte acharnée finira par être récompensée : je trouve enfin ces fameuses bd !

Le numéro un, entre mes mains, c'est dans l'un de ces fauteuils design et coloré que je m'installe. 
Dès la première page ouverte, mon corps et mon esprit se sont vu téléportés au milieu de cette fabuleuse histoire !
Quelle émotion !
Malheureusement, cela ne pouvait durer… Que ne fut pas ma déception, quand une employée, la voie vide est glacée, et venu m'informer que ça allait fermer !
Je lui fis instantanément part de mon envie d'emprunter quelques unes de ces bd, pour continuer ma lecture chez moi !
C'est ainsi qu'en grande hâte, je prend ma carte Pass'relle, pour la modique somme de 4\euro{}, et sort tout frétillant avec 4 bd de XIII sous les bras. 

Le chaos s'étant dissipé à l'arrêt de tram, je le rejoignais en ne pouvant m'empêcher de me torturer l'esprit à la recherche d'une solution pouvant me permettre de ne pas rendre ces livres en retard…
C'est à ce moment que je me suis rappelé de l'application fantastique permettant de gérer les db-thèques dont m'avait parlé l'ami Jean, quelques heures plut tôt.
Ni d'une, ni de deux, je sors mon Sony Xperia® de la poche, et cherche l'application Royal sur le \emph{market}. 

\subsection{Récupération des ISBN}

Le téléchargement terminé, le programme se lance et me demande quelle adresse mail je veux utiliser pour la synchronisation avec « mon ordinateur ».
Je ne comprend pas très bien ce qu'on me demande, mais en dessous se trouve un petit bouton « en savoir plus » qui me renseigne d'avantage sur cette chose.
Après avoir compris à quoi ça servait, j'ai décidé de coché la case « Adresse e-mail par défaut du compte Androïd » plutôt que d'en renseigner une autre. 
Une fenêtre à deux boutons se présente devant moi.
Sur le premier est indiqué « Scanner un seul ouvrage », sur le second, « Scanner un lot d'ouvrages ». 
Vue que j'ai quatre bd devant moi, c'est vers la deuxième option que je me tourne. 
Un message apparait, me demandant de viser le code barre du livre avec la caméra de mon appareil. J'appuie sur « ok », puis, j'ai l'impression que mon appareil photo s'allume. 

Cool, je prend donc en photo le code barre du livre.
Un message s'affiche à présent à l'écran, avec marqué dessus « Vérification du livre » avec une barre de progression en dessous.
Une fois la barre rempli, un une autre phrase s'affiche : « Ouvrage identifié : XIII — Le Jour du soleil noir » la remplaça.
Quelle ne fut pas ma stupéfaction quand j'ai aperçu ces quelques mots !
Seul un puissant sorcier pouvait être capable d'un tel sortilège !
Une fois mon calme revenu, je décidais de continuer à utiliser ce logiciel fabuleux, bien qu'aillant une certaine appréhension due à mon doute sur la légitimité de la magie utilisée. 
(Un sortilège tellement puissant ne peut qu'être de la magie noir).

Bref, après ce message, un autre apparait me demandant si c'est la fin du lot, ou si, au contraire, d'autres livres restent à scanner. 
J'opte donc pour le second cas, afin d'enregistrer mes trois autres bd dans le logiciel. 
Malheureusement, arrivé au dernier, la machine m'indique que l'ouvrage est introuvable et me demande donc de rentrer à la main le titre du livre. 

Une fois mes quatre bd de scannés, je choisi de terminer l'enregistrement. 
Peu de temps après, un nouveau message apparait : « Un e-mail de synchronisation a bien été envoyé sur votre boite mail, veuillez effectuer un import depuis votre ordinateur ».
« Cool alors ! » Pensais - je avec satisfaction, vivement que j'arrive chez moi, que je puisse tester tout ça !


\subsection{Utilisation de l'application sur ordinateur}
Une fois sorti du tram, c'est en grande hâte que je traverse la rue qui s'obscurcit rapidement avec la tombée de la nuit. 
Madame Bernard, la concierge était déjà rentrée chez elle, heureusement que j'avais les clefs. 
La porte ouverte, je profitais du temps de boot de Windows™ Vista® pour me faire couler un café et préparer la gamelle de Gabriel, mon petit chat roux.

Une fois Windows® démarré, je me rendis sur le site officiel de Royal pour télécharger l'application à exécuter sur l'ordinateur. 
Après cela, un installateur apparait, me demandant dans quel répertoire, l'application devait se voir installer. 
C'est vers le choix par défaut, que je me suis tourné. 
L'installation est vraiment très rapide, une fois terminée, le programme s'exécute, et je me retrouve de nouveau face à une fenêtre me demandant de choisir quelle adresse email j'allai utiliser pour la synchronisation avec le « client Androïd ». 
J'en déduis que c'est la même adresse que toute à l'heure qui doit être renseignée ici.

J'atterris ensuite sur la page centrale du programme. 
Celle - ci ressemble à une liste sur la quelle tous mes livres sont répertoriés. 
Elle est vide pour le moment, et les boutons de gestion sont bien visible sur la partie haute de la fenêtre. 
On y retrouve un bouton pour ajouter un ouvrage, pour en supprimer un, pour l'édition, et enfin pour la synchronisation. 

D'instinct, je clique sur le bouton de synchronisation.
Une autre fenêtre apparait, m'indiquant que la synchronisation d'un élément était en cour d'exécution. 
On peut voir en dessous une barre de progression, qui prend se remplie, au fur est à mesure de l'avancée des étapes : 
« Connexion au serveur mail », puis « Identification mail », après « Recherche des fichiers », et enfin « Téléchargement des ISBN ».

Après ce temps d'attente, la fenêtre disparait, et une nouvelle affiche le récapitulatif des livres importés.
Le numéro ISBN est marqué dans une colonne, à côté, est écris « Téléchargement des informations », au fur est à mesure que le temps passe, les informations s'affichent pour chaque livre, sauf celui pour le quel un message d'erreur à été émis lors de la prise de photo de l'ISBN. 
C'est pas grave, je peux toujours rentrer moi même les informations dans les champs laissé vide par défaut. 

Une fois satisfait de la présentation de livre ,je clique sur « Confirmer les informations des livres ». 
La prochaine fenêtre me demande si j'ai acheté ces ouvrages ou si je les ais empruntés dans une bibliothèque. 
J'indique donc l'emprunt.
Le programme me demande ensuite dans quelle bibliothèque je l'ai effectué. 
Vu que l'installation est toute fraiche, aucune bibliothèque n'est existante pour le moment, la liste est vide.
Je clique sur le bouton d'ajout, et une autre fenêtre s'ouvre, me demandant de renseigner le nom de la bibliothèque, son numéro de téléphone, la durée maximale de l'emprunt, et ses horaires d'ouverture. 
Je remplis tous les champs avec une grande attention, puis, valide la bibliothèque. 

De retour dans sur la page principale de Royal, la liste des ouvrages que j'ai importé est à présent marquée dans la liste. 
Nous pouvons lire à droite quelques informations sur l'ouvrage sélectionné, dont la date de retour et la bibliothèque concernée. 
Bien heureux de ce logiciel, je décide d'importer également dans le programme le livre sur le C que j'ai emprunté à l'IUT. 

\subsection {Autres fonctionnalités}

Je reprend mon bel Xperia® pour photographier le code barre du livre, et choisi cette fois le mode de scann simple, adapté pour un seul livre. 
L'appareil photo se rallume, je prend la photo du code barre, puis le nom de l'ouvrage réapparait sur l'écran.
Je confirme, et l'exportation s'effectue apparemment avec succès. 

De retour devant mon écran d'ordinateur, je ré effectue une importation. 
Encore une fois, une fenêtre s'ouvre devant moi, avec les informations auto - complétées de l'ouvrage, il ne me reste plus qu'à choisir la bibliothèque dans la quelle j'ai effectué l'emprunt, à confirmer, et le livre s'ajoute dans la liste principale !

Heureux de ma découverte et fatigué de cette longue journée, c'est vers mon lit que je me retourne pour m'enfoncer dans les songes et me laisser transporter dans un monde fantastique inconnu par le sens et la rationalité :) 

