\documentclass[etudiants]{support-iutrs}
\usepackage{pdfpages}
\usepackage{wrapfig}
\usepackage{graphicx}
\usepackage{subfig}
\usepackage{lscape}
\usepackage{longtable}
\usepackage[utf8]{inputenc}
\usepackage{array}
\usepackage{multirow}
\usepackage{color}
\usepackage{colortbl}

\infos{Benedick Steve, Hagner Kevin, Meyblum Jean}{G32}{T306AMAO}

\sujet{}
\titre{Évaluation des risques dans notre projet T3}

\begin{document}

\header

\paragraph{}
Dans ce document nous évaluerons les risques au sein de notre projet T3. 
Cette évaluation des risques nous permettra d'anticiper et d'évaluer les principaux problèmes susceptible d'arriver dans lors de notre projet.
Elle aura aussi pour but de nous permettre ne mettre en place des actions préventives afin de ne pas être confronter au différents problèmes.  

\section*{Définition du projet}

Notre projet a pour but la gestion informatique des emprunts et achats de bandes dessinées (\emph{BD}) réalisés par un particulier.
Afin de gérer la saisie et l'enregistrement des bandes dessinées nous allons modifier une application déjà existante : \emph{Royal}
\footnote{Site officiel de l'application \emph{Royal} : \url{http://royal-project.org}}.

Les modifications résident notamment dans l'ajout d'informations à stocker pour un album, avec notamment la gestion des bibliothèques dans les-quelles l'emprunt aurait pu être réalisé, et les dates de retour. 

Dans le but de faciliter l'ajout de \emph{BDs} dans Royal nous allons aussi mettre en place une application pour les téléphones Android : \emph{Royal\_Scanner}.
Cette application permettra de scanner les codes barres afin de les envoyer au client PC Royal via un email.

Il faudra donc également implémenter tout un processus de synchronisation entre les deux applications. 

\section*{Explication du tableau}

\begin{description}

\item[Détection]
Date de détection du risque.

\item[Apparition]
Correspond à la date où le risque est apparu. Cette date n'est pas toujours indiqué, dans le cas où le risque n'est pas encore survenu.

\item[Titre]
Correspond aux titres des différents risques classés par type.

\item[Probabilité]
Probabilité de survenance du risque. "1" Correspond à une probabilité de survenance situé entre 1\% et 20\%. "2" correspond à un probabilité entre 20\% et 80\% et "5" indique une probabilité supérieur à 80\%.

\item[Gravité]
Gravité du risque s'il survenait. La gravité est noté de la même manière que la probabilité. avec "1" pour une gravité faible, "2" pour une gravité moyenne et "5" pour une gravité forte pouvant être fatale pour la bonne réalisation du projet.

\item[Indetectabilité]
Niveau de détection du risque. Il est noté par les notes "1", "2", ou "4". On définit ici si le risque peut être perçu facilement en avance ou non.

\item[Niveau du risque]
Le niveau du risque est une note globale pour un risque donné correspondant à la multiplication des trois colonnes précédentes. Elle correspond à un pourcentage d'importance du risque.

\item[Action préventives]
Actions pouvant être mises en place pour éviter l'apparition du risque.

\item[Coûts]
Indique le coûts des différentes actions préventive que ce soit en volume horaire ou en prix. Dans le cadre de ce projet il ne s'agit que de volume horaire.  
\end{description}

\begin{landscape}

\begin{longtable}{||l|l|c|c|c|c|c|c|m{6.5cm}|c||}
\caption{Liste des risques aux quels nous pouvons être soumis}
\label{Un grand tableau} \\
\hline 
	\textbf{No} &
	\textbf{Risques} & 
	\textbf{Détection} & 
	\textbf{Apparition} & 
	\textbf{Proba} & 
	\textbf{Gvt} & 
	\textbf{Indetect.} & 
	\textbf{Nv rsq} & 
	\textbf{Action préventive} &
	\textbf{Cout} \\
\hline \hline
\endhead

	\multicolumn{10}{||c||}{\rowcolor{gray} \color{white}Paramètres généraux} \\
\hline

	\multirow{2}*{1} &
	\multirow{2}*{Définition imprécise d'un projet }&
	\multirow{2}*{03\slash{}09\slash{}2011} &
	\multirow{2}*{10\slash{}09\slash{}2011} &
	\multirow{2}*{1}&
	\multirow{2}*{1}&
	\multirow{2}*{1}&
	\multirow{2}*{1}&
	Réunion avec le maitre d'ouvrage. &
	6h\\
	\cline{9-10}
	&&&&&&&&Multiples livrables. &\\
\hline

	2 &
	Taille du projet et durée &
	15\slash{}09\slash{}2011 &
	&1&2&2&4&
	Estimation MS Project. &
	5h \\
\hline

	\multirow{2}*{3} &
	\multirow{2}*{Complexité des projets} &
	\multirow{2}*{12\slash{}09\slash{}2011} &
	\multirow{2}*{12\slash{}09\slash{}2011} &
	\multirow{2}*{2} &
	\multirow{2}*{5} &
	\multirow{2}*{1} &
	\multirow{2}*{10} &
	Réunion avec les développeurs du projet initial. &
	3h \\
	\cline{9-10}
	&&&&&&&&Analyse en profondeur du code existant. &
	16h \\
\hline 

	4 &
	Indépendance des projets &
	15\slash{}09\slash{}2011 &
	&5&2&1&10&
	Travail collaboratif, dépôt commun, ordre de développement. &
	4h \\
\hline 

	5 &
	Int erface \emph{(Scanner)} &
	10\slash{}10\slash{}2011 &
	10\slash{}10\slash{}2011 &
	2&5&1&10&
	Intégration de l'interface à l'application. &
	16h \\ 
\hline 

	6 &
	Interface \emph{(BDD ISBN)} &
	20\slash{}09\slash{}2011 &&
	1&5&1&5&
	Choix d'une interface simple ou complexe. &
	8h \\
\hline

	7 &
	Interface \emph{(Email)} &
	22\slash{}09\slash{}2011 &&
	1&5&1&5&
	Mise en place d'une norme pour les emails. &
	4h \\
\hline

	\multirow{2}*{8} &
	\multirow{2}*{Sophistication} &
	\multirow{2}*{10\slash{}09\slash{}2011} &
	\multirow{2}*{10\slash{}09\slash{}2011} &
	\multirow{2}*{5}&
	\multirow{2}*{1}&
	\multirow{2}*{2}&
	\multirow{2}*{10}&
	Réunion avec le maitre d'ouvrage. & 
	6h \\
	\cline{9-10}
	&&&&&&&&Mise au courant permanent des phases de développement. &
	4h \\
\hline 

	\multicolumn{10}{||c||}{\rowcolor{gray} \color{white}Paramètres liés aux utilisateurs} \\
\hline

	\multirow{2}*{9} &
	\multirow{2}*{Compétances} &
	\multirow{2}*{10\slash{}09\slash{}2011} &&
	\multirow{2}*{5} &
	\multirow{2}*{2} &
	\multirow{2}*{1} &
	\multirow{2}*{10} &
	Manuel utilisateur. &
	20h \\
	\cline{9-10}
	&&&&&&&&Appliquer des règles d'ergonomie. &
	7h \\
\hline

	\multicolumn{10}{||c||}{\rowcolor{gray} \color{white}Paramètres liés à la maitrise d'ouvrage} \\
\hline

	10 &
	Décision &
	07\slash{}11\slash{}2011 &&
	1&1&1&1&
	Proposition de solution. &\\
\hline

	\multirow{2}*{11} &
	\multirow{2}*{Coordination maitrise d'ouvrage} &
	\multirow{2}*{15\slash{}09\slash{}2011} &
	\multirow{2}*{15\slash{}09\slash{}2011} &
	\multirow{2}*{2} &
	\multirow{2}*{2} &
	\multirow{2}*{1} &
	\multirow{2}*{4} &
	Privilégier les rapports directs. &\\
	\cline{9-10}
	&&&&&&&&Mise en place de réunions régulières. &
	6h\\
\hline

\newpage
	\multicolumn{10}{||c||}{\rowcolor{gray} \color{white}Paramètres liées à la maitrise d'œuvre} \\
\hline

	12 &
	Maîtrise des demandes &
	07\slash{}11\slash{}2011 &&
	1&2&4&8&
	Ne pas accepter les demandes non décrites dans le \emph{CDC}. & \\
\hline

	\multirow{3}*{12} & 
	\multirow{3}*{Répartition des charges} &
	\multirow{3}*{01\slash{}10\slash{}2011} &&
	\multirow{3}*{2} &
	\multirow{3}*{2} &
	\multirow{3}*{2} &
	\multirow{3}*{8} &
	Estimation \emph{MS Project}. &
	5h\\
	\cline{9-10}
	&&&&&&&&Analyse des compétences de chacun. & 
	1h \\
	\cline{9-10}
	&&&&&&&&Réunion entre les développeurs. & 
	10h \\
\hline 

	13 &
	Méthodes \slash{} Normes &
	10\slash{}10\slash{}2011 &&
	1&1&1&1&
	Utilisation de normes de codage précises. &
	1h \\
\hline


	\multirow{2}*{14} &
	\multirow{2}*{Équipe projet} &
	\multirow{2}*{03\slash{}09\slash{}2011} && 
	\multirow{2}*{2} &
	\multirow{2}*{5} &
	\multirow{2}*{1} &
	\multirow{2}*{10} &
	Choix des coéquipiers. & \\
	\cline{9-10}
	&&&&&&&&Répartition des tâches en fonction des compétences et affinités. & 
	1h\\ 
\hline

	15 &
	Suivi &
	10\slash{}10\slash{}2011 &&
	5&2&1&10&
	Publier l'avancement du travail en permanence. &
	\\
\hline

	16 &
	Gestion des incidents \slash{} dérives &
	15\slash{}10\slash{}2011 &&
	2&2&2&8&
	Tenir un journal des incidents. &
	2h \\
\hline

\end{longtable}

\end{landscape}

\section{Définition des différents risques}
\subsection{Définition imprécise d'un projet}

Le premier risque rencontré dans notre projet était la définition imprécise du projet. 
En effet sans une définition précise de notre projet, il y a un grand risque que notre travaille ne corresponde pas réellement à la demande notre tuteur. 
Afin de prévoir ce risque une mise en place de réunion régulière lors de l’élaboration du cahier des charges ainsi que celle du dossier d'analyse pour bien définir le projet.
De plus le nombre de livrables à été augmenté par rapport à celui prévu initialement.

\subsection{Taille du projet}

Le risque d'avoir un projet trop conséquent est fortement lié à la définition imprécise du projet. Ce risque, s'il survient peut entrainer à une restitution incomplète du projet, voir à un projet non fonctionnel. Pour palier à ce risque, il est nécéssaire d'avoir un estimation précise du temps nécessaire à la réalisation des différentes fonctionnalités du projet.

\subsection{Compléxité des projets}

Dans notre cas précis, nous devons continuer le développement d'un projet éxistant. Le risque dans ce cas ci est de nous retrouver face à un projet trop complexe que se soit pour la compréhension mais aussi pour la modification. Pour éviter ce problème il est nécessaire de prévoir du temps pour l'analyse du code éxistant et prendre rendez-vous avec l'ancienne équipe de développement pour pouvoir des informations suplémentaires. 

\subsection{Interdépendance des projets}

Etant donné que notre projet peut se développer sur plusieurs fronts à la fois, il est nécessaire d'ordonné le dévellopement des différentes parties, pour ne pas que le développement d'une partie dépende d'une fonctionnalité non développée. C'est pour cela qu'il est nécessaire d'utiliser un système de dépot pour mettre facilement les fichiers en commun.

\subsection{Interface \emph{(Scanner)}}

L'interface \emph{Scanner} est une application indépendante importante pour le projet car elle permet de capturer un code barre grâce à l'appareil photo du téléphone mobile. Dans ce cas-ci, cette application se télécharge au lancement de notre application et communique avec elle le code barre capturé. Le risque est que la version du Scanner diffère et devienne incompatible avec notre application. C'est pour cela qu'il faut intégré directement l'interface Scanner dans notre application.

\subsection{Interface \emph{BDD ISBN}}

Notre application à besoin de communiquer avec une base de données distante afin de récupérer des informations concernant des bandes déssinés. Ce genre de base de données éxiste et son disponible sur le réseau. Plusieurs problématique se pose cependant. Dans un premier temps, il faut choisir la base de données la plus complète, pour cela il est nécessaire de prévoir du temps pour tester ces bases afin de voir laquelle d'entre elle est la plus fournie. Dans un second temps il faut choisir celle qui permet le plus d'accès pour un temps donné. Mais le plus important est l'accès à la base. Il faut choisir un système d'accès simple et complet facilement intégrable au projet et qui ne changera pas dans le temps.

\subsection{Interface \emph{Email}}

Il est nécessaire de prévenir ce risque, en effet, ici les deux applications à développer communiqueront ensemble par un échange d'e-mails. C'est pourquoi il faut mettre en place à l'avance les normes de structure des e-mails afin que les deux parties  puissent communiquer et se comprendre.

\subsection{Sophistication}

Les outils étant très variés, et les possibilités d'améliorations vastes, il est nécessaire de définir, notamment grâce au cahier des charges, ce qu'il faut développer en priorité, qu'elle sont les améliorations possibles, souhaitées ou non souhaitées par le maître d'ouvrage.

\subsection{Compétences}

Ici, il s'agit des compétences de l'utilisateur. Effectivement, un utilisateur lambda doit être capable d'utiliser et de comprendre le fonctionnement du logiciel. Pour éviter tout risques de mauvaise ou de non compréhension, l'établissement d'une charte ergonomique et graphique doit être appliquée. De plus, dans le cas où l'utilisateur à besoin d'aide, un manuel d'utilisateur serait un bon outil pour pallier ce problème.

\subsection{Décision}

Dans le cas où notre tuteur ne serait pas nous diriger, ou qu'il y aurait un problème décisionnel. Nous devons être en mesure de lui proposer diverses solutions possible à mettre en oeuvre afin de répondre aux demandes de l'utilisateur. 


\subsection{Coordination maîtrise d'ouvrage – maîtrise d’œuvre}

Afin d'éviter tout égarement, non respect du cahier des charges ou encore la mauvaise compréhension d'une demande de la maîtrise d'ouvrage, il faudra veillez à la mise en place de réunion régulières entre la maîtrise d'ouvrage et la maîtrise d’œuvre. 
De plus des rapport directs seront privilégié à l'envoi d'e-mail lorsque cela sera possible.    

\subsection{Maitrise des demandes}

Pour maitriser le plus possible l'évolution des travaux, on a décidé de se plier uniquement à la résolution des objectifs prévus dans le cahier des charges, et de ne pas dévier en fonction des envies ou remarques effectués plus tard.
Bien sûr, si une modification mineur est à effectuer, sans incidence sur le reste du projet, on pourra toujours essayé de voir si c'est faisable sans casser tout le reste.

\subsection{Répartition des charges}

La répartition des charges c'est tout d'abord effectuée en fonction des motivations et centres d'intérêt de chacun. 
Après nous avons détaillé l'analyse grâce à \emph{MS Project}.
Pour limiter les risques de mauvaise répartition, nous effectuons une réunion hebdomadaire pendant la quelle chaque personne exprime ce qu'elle a fait durant la semaine et si elle s'estime « chargée ». 

\subsection{Méthode \slash{} Normes}

Afin de rendre le projet le plus homogène possible, on a décidé de nous baser sur la même norme de rédaction des codes sources pour éviter que les habitudes de chacun (dans les nom des variables ou dans l'indentation par exemple) ne gène à la compréhension de celui-ci par les autres.
Comme nous travaillons sur un projet déjà existant, nous nous basons sur la typographie de celui-ci ainsi que sur quelques normes déjà définies par les anciens contributeurs 
\footnote{Normes d'écriture déjà en place dans \emph{Royal} : \url{http://trac.royal-project.org/wiki/BeauCode}}. 

\subsection{Equipe projet}

Le développement des différentes fonctionnalités se fait par affinités et par compétences, pour augmenter notre rendemenent et pour lutter contre la baisse de moral ou de motivation. De plus, l'équipe doit restée soudée, pour cela il est nécessaire de bien choisir ses coéquipiers et de maintenire des rencontres afin de discuter de notre état de satisfaction face au projet et avec nos collègues de travail.

\subsection{Suivi}

Le suivi de notre projet s'effectue rapidement via notre dépôt \emph{Github} 
\footnote{Adresse du dépôt \emph{Github} distant : \url{https://github.com/spydemon/Royal_/}}.
Les taches en cours peuvent aussi être visibles en permance par l'equipe de développement pour que tout le monde sache sur quoi les autres travailles et ainsi éviter que deux personnes face la même chose. 


\end{document}
