\documentclass[etudiants]{support-iutrs}
\usepackage{pdfpages}
\usepackage{wrapfig}
\usepackage{graphicx}
\usepackage{subfig}
\usepackage{lscape}
\usepackage{longtable}
\usepackage{array}

\infos{Benedick Steve, Meyblum Jean, Hagner Kevin}{G32}{T306AMAO}

\sujet{}
\titre{Évaluation des risques dans notre projet T3}

\begin{document}

\header
\section*{Définition du projet}

Notre projet a pour but la gestion informatique des emprunts et achats de bandes dessinées (\emph{BD}) réalisés par un particulier.
Afin de gérer la saisie et l’enregistrement des bandes dessinées nous allons modifier une application déjà existante : \emph{Royal}
\footnote{Site officiel de l'application \emph{Royal} : \url{http://royal-project.org}}.

Les modifications résident notamment dans l'ajout d'informations à stocker pour un album, avec notamment la gestion des bibliothèques dans les-quelles l'emprunt aurait pu être réalisé, et les dates de retour. 

Dans le but de faciliter l'ajout de \emph{BDs} dans Royal nous allons aussi mettre en place une application pour les téléphones Android : \emph{Royal\_Scanner}.
Cette application permettra de scanner les codes barres afin de les envoyer au client PC Royal via un email.

Il faudra donc également implémenter tout un processus de synchronisation entre les deux applications. 

\section*{Explication du tableau}

\subsection{Titre}
Correspond aux titres des différents risques classés par type.

\subsection{Probabilité}
Probabilité de survenance du risque. "1" Correspond à une probabilité de survenance situé entre 1\% et 20\%. "2" correspond à un probabilité entre 20\% et 80\% et "5" indique une probabilité supérieur à 80%.

\subsection{Gravité}
Gravité du risque s'il survenait. La gravité est noté de la même manière que la probabilité. avec "1" pour une gravité faible, "2" pour une gravité moyenne et "5" pour une gravité forte pouvant être fatale pour la bonne réalisation du projet.

\subsection{Indetectabilité}
Niveau de détection du risque. Il est noté par les notes "1", "2", ou "4". On définit ici si le risque peut être perçu facilement en avance ou non.

\subsection{Niveau du risque}
Le niveau du risque est une note globale pour un risque donné correspondant à la multiplication des trois colonnes précédentes. Elle correspond à un pourcentage d'importance du risque.

\subsection{Actions préventives}

\begin{landscape}

%\begin{longtable}{||l|m{2cm}|m{2cm}|m{2cm}|m{2cm}|p{9cm}|m{2cm}||}
\begin{longtable}{||l|c|c|c|c|p{9cm}|c||}
\caption{Liste des risques aux quels nous pouvons être soumis}
\label{Un grand tableau} \\
\hline \textbf{Titre} & \textbf{Proba} & \textbf{Gravité} & \textbf{Indétect.} & \textbf{Niveau de risque} & \textbf{Actions préventives} & \textbf{Coût} \\
\hline \hline
\endhead
Voici un super tableau &
	60 & 
	10 &
	12 & 
	25 &
	Un truc bien long pour explique le risque et voir ce que ça va donner. Normalement c'est bon, mais après on ne sais jamais. On verra bien quoi, il faut dire qu'on a pas mal de place quand même. &
	100 \\
\hline
Une deuxième entrée maintenant & 
	10 &
	25 &
	12 &
	40 & 
	Alors là, c'est un peu bizarre comme truc. En fait c'est pas quelque chose de très complèxe mais on ne sait jamais ce qui peut arrivé. Après c'est pas comme si c'était la fin du monde non plus, on devrait survire ^^ &
	150 \\
\hline
Encore un mégatruc &
	10 &
	12 &
	14 &
	16 &
	Il va bientôt pleuvoir dehord je crois. Je suis tenté de dire que c'est triste, mais d'un côté, la pluie c'est bien aussi. Ne serait-ce que pour la nappe phréatique qui en a certainement besoin. Et puis, on peut dire ce qu'on veut, mais la pluie a aussi son charme. &
	100 \\
\hline 
Encore un mégatruc &
	10 &
	12 &
	14 &
	16 &
	Il va bientôt pleuvoir dehord je crois. Je suis tenté de dire que c'est triste, mais d'un côté, la pluie c'est bien aussi. Ne serait-ce que pour la nappe phréatique qui en a certainement besoin. Et puis, on peut dire ce qu'on veut, mais la pluie a aussi son charme. &
	100 \\
\hline 
Encore un mégatruc &
	10 &
	12 &
	14 &
	16 &
	Il va bientôt pleuvoir dehord je crois. Je suis tenté de dire que c'est triste, mais d'un côté, la pluie c'est bien aussi. Ne serait-ce que pour la nappe phréatique qui en a certainement besoin. Et puis, on peut dire ce qu'on veut, mais la pluie a aussi son charme. &
	100 \\
\hline 
Encore un mégatruc &
	10 &
	12 &
	14 &
	16 &
	Il va bientôt pleuvoir dehord je crois. Je suis tenté de dire que c'est triste, mais d'un côté, la pluie c'est bien aussi. Ne serait-ce que pour la nappe phréatique qui en a certainement besoin. Et puis, on peut dire ce qu'on veut, mais la pluie a aussi son charme. &
	100 \\
\hline 
Encore un mégatruc &
	10 &
	12 &
	14 &
	16 &
	Il va bientôt pleuvoir dehord je crois. Je suis tenté de dire que c'est triste, mais d'un côté, la pluie c'est bien aussi. Ne serait-ce que pour la nappe phréatique qui en a certainement besoin. Et puis, on peut dire ce qu'on veut, mais la pluie a aussi son charme. &
	100 \\
\hline 
Encore un mégatruc &
	10 &
	12 &
	14 &
	16 &
	Il va bientôt pleuvoir dehord je crois. Je suis tenté de dire que c'est triste, mais d'un côté, la pluie c'est bien aussi. Ne serait-ce que pour la nappe phréatique qui en a certainement besoin. Et puis, on peut dire ce qu'on veut, mais la pluie a aussi son charme. &
	100 \\
\hline 
C'est magifique &
Dis donc & 
:) \\ 
\hline
\end{longtable}

\end{landscape}

\end{document}
