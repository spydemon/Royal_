\documentclass[etudiants]{support-iutrs}
\usepackage{pdfpages}
\usepackage{wrapfig}
\usepackage{graphicx}
\usepackage{subfig}
\usepackage{lscape}
\usepackage{longtable}
\usepackage[utf8]{inputenc}
\usepackage{array}
\usepackage{multirow}
\usepackage{color}
\usepackage{colortbl}

\infos{Benedick Steve, Meyblum Jean, Hagner Kevin}{G32}{T306AMAO}

\sujet{}
\titre{Évaluation des risques dans notre projet T3}

\begin{document}

\header
\section*{Définition du projet}

Notre projet a pour but la gestion informatique des emprunts et achats de bandes dessinées (\emph{BD}) réalisés par un particulier.
Afin de gérer la saisie et l'enregistrement des bandes dessinées nous allons modifier une application déjà existante : \emph{Royal}
\footnote{Site officiel de l'application \emph{Royal} : \url{http://royal-project.org}}.

Les modifications résident notamment dans l'ajout d'informations à stocker pour un album, avec notamment la gestion des bibliothèques dans les-quelles l'emprunt aurait pu être réalisé, et les dates de retour. 

Dans le but de faciliter l'ajout de \emph{BDs} dans Royal nous allons aussi mettre en place une application pour les téléphones Android : \emph{Royal\_Scanner}.
Cette application permettra de scanner les codes barres afin de les envoyer au client PC Royal via un email.

Il faudra donc également implémenter tout un processus de synchronisation entre les deux applications. 

\section*{Explication du tableau}

\subsection{Détection}
Date de détection du risque.

\subsection{Apparition}
Correspond à la date où le risque est apparu. Cette date n'est pas toujours indiqué, dans le cas où le risque n'est pas encore survenu.

\subsection{Titre}
Correspond aux titres des différents risques classés par type.

\subsection{Probabilité}
Probabilité de survenance du risque. "1" Correspond à une probabilité de survenance situé entre 1\% et 20\%. "2" correspond à un probabilité entre 20\% et 80\% et "5" indique une probabilité supérieur à 80%.

\subsection{Gravité}
Gravité du risque s'il survenait. La gravité est noté de la même manière que la probabilité. avec "1" pour une gravité faible, "2" pour une gravité moyenne et "5" pour une gravité forte pouvant être fatale pour la bonne réalisation du projet.

\subsection{Indetectabilité}
Niveau de détection du risque. Il est noté par les notes "1", "2", ou "4". On définit ici si le risque peut être perçu facilement en avance ou non.

\subsection{Niveau du risque}
Le niveau du risque est une note globale pour un risque donné correspondant à la multiplication des trois colonnes précédentes. Elle correspond à un pourcentage d'importance du risque.

\subsection{Actions préventives}
Actions pouvant être mises en place pour éviter l'apparition du risque.

\begin{landscape}

\begin{longtable}{||l|c|c|c|c|c|c|m{7cm}||}
\caption{Liste des risques aux quels nous pouvons être soumis}
\label{Un grand tableau} \\
\hline 
	\textbf{Risques} & 
	\textbf{Détection} & 
	\textbf{Apparition} & 
	\textbf{Proba} & 
	\textbf{Gravité} & 
	\textbf{Indetect.} & 
	\textbf{Niveau risque} & 
	\textbf{Action préventive} \\
\hline \hline
\endhead

	\multicolumn{8}{||c||}{\rowcolor{gray} \color{white}Paramètres généraux} \\
\hline

	\multirow{2}*{Définition imprécise d'un projet }&
	\multirow{2}*{Début Septembre } &
	\multirow{2}*{Début Septembre} &
	\multirow{2}*{1}&
	\multirow{2}*{1}&
	\multirow{2}*{1}&
	\multirow{2}*{1}&
	Réunion avec le maitre d'ouvrage. \\
	\cline{8-8}
	&&&&&&&Multiples livrables. \\
\hline

	Taille du projet et durée &
	Début Septembre &
	&1&2&2&4&
	Estimation MS Project. \\
\hline

	\multirow{2}*{Complexité des projets} &
	\multirow{2}*{Début Septembre } &
	\multirow{2}*{Début Septembre} &
	\multirow{2}*{2} &
	\multirow{2}*{5} &
	\multirow{2}*{1} &
	\multirow{2}*{10} &
	Réunion avec les développeurs du projet initial. \\
	\cline{8-8}
	&&&&&&&Analyse en profondeur du code existant. \\
\hline 

	Indépendance des projets &
	Fin Septembre &
	&5&2&1&10&
	Travail collaboratif, dépôt commun, ordre de développement. \\
\hline 

	Interface \emph{(Scanner)} &
	Début Octobre &
	Milieu Octobre &
	2&5&1&10&
	Intégration de l'interface à l'application. \\ 
\hline 

	Interface \emph{(BDD ISBN)} &
	Début Octobre &&
	1&5&1&5&
	Choix d'une interface simple ou complexe. \\
\hline

	Interface \emph{(Email)} &
	Début Octobre &&
	1&5&1&5&
	Mise en place d'une norme pour les emails. \\
\hline

	\multirow{2}*{Sophistication} &
	\multirow{2}*{Début Septembre} &
	\multirow{2}*{Début Septembre} &
	\multirow{2}*{5}&
	\multirow{2}*{1}&
	\multirow{2}*{2}&
	\multirow{2}*{10}&
	Réunion avec le maitre d'ouvrage. \\
	\cline{8-8}
	&&&&&&&Mise au courant permanent des phases de développement. \\
\hline 

	\multicolumn{8}{||c||}{\rowcolor{gray} \color{white}Paramètres liés aux utilisateurs} \\
\hline

	\multirow{2}*{Compétances} &
	\multirow{2}*{Mi Septembre} &&
	\multirow{2}*{5} &
	\multirow{2}*{2} &
	\multirow{2}*{1} &
	\multirow{2}*{10} &
	Manuel utilisateur. \\
	\cline{8-8}
	&&&&&&&Appliquer des règles d'ergonomie. \\
\hline

	\multicolumn{8}{||c||}{\rowcolor{gray} \color{white}Paramètres liés à la maitrise d'ouvrage} \\
\hline

	Décision &
	Début Septembre &&
	1&1&1&1&
	Proposition de solution. \\
\hline

	\multirow{2}*{Coordination maitrise d'ouvrage} &
	\multirow{2}*{Début Septembre} &
	\multirow{2}*{Début novembre} &
	\multirow{2}*{2} &
	\multirow{2}*{2} &
	\multirow{2}*{1} &
	\multirow{2}*{4} &
	Privilégier les rapports directs. \\
	\cline{8-8}
	&&&&&&&Mise en place de réunions régulières. \\
\hline

\newpage
	\multicolumn{8}{||c||}{\rowcolor{gray} \color{white}Paramètres liées à la maitrise d'œuvre} \\
\hline

	Maîtrise des demandes &
	Fin Octobre &&
	1&2&4&8&
	Ne pas accepter les demandes non décrites dans le \emph{CDC}. \\
\hline

	\multirow{3}*{Répartition des charges} &
	\multirow{3}*{Début Octobre} &&
	\multirow{3}*{2} &
	\multirow{3}*{2} &
	\multirow{3}*{2} &
	\multirow{3}*{8} &
	Estimation \emph{MS Project}. \\
	\cline{8-8}
	&&&&&&&Analyse des compétences de chacun. \\
	\cline{8-8}
	&&&&&&&Réunion entre les développeurs. \\
\hline 

	Méthodes \slash{} Normes &
	Mi Septembre &&
	1&1&1&1&
	Utilisation de normes de codage précises. \\
\hline


	\multirow{2}*{Équipe projet} &
	\multirow{2}*{Début Septembre} && 
	\multirow{2}*{2} &
	\multirow{2}*{5} &
	\multirow{2}*{1} &
	\multirow{2}*{10} &
	Choix des coéquipiers. \\
	\cline{8-8}
	&&&&&&&Répartition des tâches en fonction des compétences et affinités. \\ 
\hline

	Suivi &
	Mi Septembre &&
	5&2&1&10&
	Publier l'avancement du travail en permanence. \\
\hline

	Gestion des incidents \slash{} dérives &
	Début Novembre&&
	2&2&2&8&
	Tenir un journal des incidents. \\
\hline

\end{longtable}

\end{landscape}

\section{Définition des différents risques}
\subsection{Définition imprécise d'un projet}

Le premier risque rencontré dans notre projet était la définition imprécise du Projet. 
En effet sans une définition précise de notre projet il y avait un grand risque que notre travaille ne corresponde pas réellement à la demande notre tuteur. 
Afin de répondre à ce risque une mise en place de réunion régulière lors de l’élaboration du cahier des charges ainsi que celle du dossier d'analyse pour bien définir le projet.
De plus le nombres de livrable à été augmenté par rapport a celui prévu initialement.

\end{document}
